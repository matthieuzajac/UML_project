\documentclass[12pt,a4paper]{article}
\usepackage[utf8]{inputenc}
\usepackage[french]{babel}
\usepackage[T1]{fontenc}
\usepackage{geometry}
\usepackage{graphicx}
\usepackage{xcolor}
\usepackage{hyperref}
\usepackage{enumitem}
\usepackage{booktabs}
\usepackage{fancyhdr}
\usepackage{float}
\usepackage{amsmath}
\usepackage{multicol}
\usepackage{listings}

\newcommand{\hsp}{\hspace{20pt}}
\newcommand{\HRule}{\rule{\linewidth}{0.5mm}}

\geometry{margin=2.5cm}
\pagestyle{fancy}
\fancyhf{}
\rhead{\thepage}
\lhead{Rapport UML - Plateforme Webnovel}



\begin{document}
\begin{titlepage}
\title{\textbf{Rapport de Conception UML}\\
\large Système de Publication de Webnovels}
\author{}
\date{\today}

  \begin{sffamily}
  \begin{center}

    % Upper part of the page. The '~' is needed because \\
    % only works if a paragraph has started.
    \includegraphics[scale=0.3]{Mines-Saint-Etienne-IMT-Logo-Hor-RVB-1000x380-1-4.png}~\\[1.5cm]

    \textsc{\LARGE École Nationale Supérieure
des Mines de Saint-Étienne}\\[2cm]

    \textsc{\Large Rapport de projet}\\[1.5cm]

    % Title
    \HRule \\[0.4cm]
    { \huge \bfseries Projet Modélisation UML\\[0.4cm] }

    \HRule \\[1 cm]

    \LARGE Site de publication de nouvelles en ligne\\[6cm]

    % Author and supervisor
    \begin{minipage}{0.4\textwidth}
      \begin{flushleft} \large
        Matthieu Zajac\\
        William Qiu\\
        Yueyang Wang\\
        EI24\\
      \end{flushleft}
    \end{minipage}
     \begin{minipage}{0.4\textwidth}
      \begin{flushright} \large
        \emph{Professeurs :} Florent Duchiron\\
      \end{flushright}
    \end{minipage}
    

    \vfill

    % Bottom of the page
    {\large Semestre 7}

  \end{center}
  \end{sffamily}
\end{titlepage}

\newpage


\maketitle
\tableofcontents
\newpage

\section{Introduction}

Ce projet présente la modélisation UML complète d'une plateforme de publication et de lecture de webnovels, inspirée de Royal Road. Le système permet aux auteurs de publier leurs romans chapitres par chapitres, aux lecteurs de découvrir et suivre du contenu, et aux modérateurs d'assurer la qualité de la plateforme. Cette documentation suit rigoureusement la norme UML et couvre les aspects statiques et dynamiques du système.

\subsection{Objectifs du système}

Le système vise à créer une plateforme complète où :
\begin{itemize}[noitemsep]
    \item Les \textbf{invités} peuvent explorer et lire du contenu public sans authentification
    \item Les \textbf{utilisateurs authentifiés} gèrent leur profil et reçoivent des notifications
    \item Les \textbf{lecteurs} peuvent créer une bibliothèque personnelle, s'abonner aux romans et publier des critiques
    \item Les \textbf{auteurs} (qui héritent des capacités de lecteur) peuvent créer, publier et gérer leurs romans
    \item Les \textbf{modérateurs} peuvent maintenir la qualité du contenu et gérer les utilisateurs problématiques
\end{itemize}

\subsection{Choix technologiques justifiés}

\textbf{PlantUML} a été retenu comme outil de modélisation pour sa syntaxe textuelle permettant le versionnage via Git, sa conformité stricte à la norme UML, et sa capacité à générer automatiquement des diagrammes à partir de définitions textuelles. Cette approche favorise la collaboration et la traçabilité des modifications.

\section{Vue Statique du Système}

\subsection{Diagramme de Cas d'Utilisation}

\begin{figure}[H]
    \centering 
    \includegraphics[width=0.8\linewidth]{usecase.png}
    \caption{Diagramme de cas d'utilisation}
    \label{fig:placeholder}
\end{figure}

Ce diagramme définit les frontières du système et identifie quatre acteurs principaux organisés par héritage : \texttt{Guest}, \texttt{Authenticated\_User}, \texttt{Reader}, \texttt{Author}, et \texttt{Moderator}. L'héritage reflète l'accumulation de droits : un auteur hérite des capacités d'un lecteur (lecture, critiques), qui hérite lui-même d'un utilisateur authentifié (profil, notifications).

Les cas d'utilisation sont regroupés en cinq packages fonctionnels : accès invité (exploration, recherche, lecture publique, inscription), gestion de compte (authentification, profil, notifications, signalement), fonctionnalités lecteur (bibliothèque, abonnements, progression, critiques), fonctionnalités auteur (création de roman, publication de chapitres, analyse de statistiques), et outils de modération (modération de contenu, gestion des utilisateurs, traitement des signalements). Les relations \texttt{<<include>>} structurent les cas d'utilisation composites tandis que \texttt{<<extend>>} modélise les extensions optionnelles comme le signalement de contenu.

\subsection{Diagramme de Classes}

\begin{figure}[H]
    \centering
    \includegraphics[width=1\linewidth]{class diagram.png}
    \caption{Diagramme de Classes}
    \label{fig:placeholder}
\end{figure}

Le diagramme de classes constitue le coeur du modèle. La classe abstraite \texttt{Authenticated\_user} centralise l'authentification et la gestion de profil, avec des attributs protégés (\texttt{id}, \texttt{username}, \texttt{email}, \texttt{password\_hash}) et des indicateurs de modération (\texttt{is\_banned}, \texttt{is\_shadowbanned}).

\textbf{Choix de conception majeur :} l'héritage \texttt{Authenticated\_user $\leftarrow$ Reader $\leftarrow$ Author} reflète le fait qu'un auteur est toujours un lecteur (peut lire et commenter d'autres œuvres), mais un lecteur n'est pas nécessairement auteur. \texttt{Moderator} hérite directement de \texttt{Authenticated\_user} car il possède des privilèges distincts.

Les entités centrales \texttt{Novel} et \texttt{Chapter} sont liées par composition (diamant noir) : la suppression d'un roman entraîne celle de tous ses chapitres. \texttt{Review} utilise l'agrégation (diamant blanc) car elle dépend du lecteur et du roman mais possède un cycle de vie partiellement indépendant.

Quatre énumérations encapsulent les états critiques : \texttt{NovelStatus} (DRAFT, ACTIVE, HIATUS, COMPLETED, SHADOWBANNED, DELETED), \texttt{NotificationType}, \texttt{ReadingStatus}, et \texttt{ReportStatus}, assurant la cohérence et la validation des transitions d'états.


\subsection{Diagramme de Composants}

\begin{figure}[H]
    \centering
    \includegraphics[width=1\linewidth]{component diagram.png}
    \caption{Diagramme de composant}
    \label{fig:placeholder}
\end{figure}

Le diagramme de composants illustre l'architecture technique en couches avec dépendances explicites. La couche frontend (Web UI, Mobile App) communique via une API Gateway (REST API, WebSocket Server) avec neuf services applicatifs spécialisés : Authentication, User, Novel, Chapter, Review, Notification, Search, Moderation, et Analytics.

Ces services s'appuient sur des composants de logique métier (User Management, Content Management, Library Management, etc.) qui utilisent les services d'infrastructure (Email Service, File Storage, Cache Service, Anti-Spam Filter, Search Engine).

\textbf{Choix architectural :} la séparation en microservices facilite la scalabilité horizontale et l'isolation des pannes. L'anti-spam est intégré au niveau infrastructure pour protéger tous les points d'entrée. Le cache (Redis) et l'index de recherche (Elasticsearch) sont découplés de la base relationnelle principale pour optimiser les performances.

\subsection{Diagramme de Déploiement}

\begin{figure}[H]
    \centering
    \includegraphics[width=1\linewidth]{deployment diagram.png}
    \caption{Diagramme de déploiement}
    \label{fig:placeholder}
\end{figure}

Le diagramme de déploiement illustre la distribution physique des composants sur l'infrastructure serveur et les protocoles de communication sécurisés utilisés entre les nœuds.

\textbf{Analyse de l'infrastructure :}
\begin{itemize}
    \item \textbf{Communication Client-Serveur :} Les échanges sont sécurisés via \texttt{HTTPS} pour les requêtes REST et \texttt{WSS} (WebSocket Secure) pour les mises à jour en temps réel (notifications, chats).
    \item \textbf{Passerelle Applicative :} L'API Gateway fait office de point d'entrée unique, gérant l'authentification et le \textit{rate limiting} avant de rediriger le trafic vers les services de logique métier.
    \item \textbf{Traitement Asynchrone :} Le système sépare les requêtes utilisateur immédiates des tâches lourdes (envoi d'emails, calculs statistiques, traitement d'images) via des \texttt{Background Workers} et des files d'attente de jobs (\textit{Queue jobs}).
    \item \textbf{Stratégie de Stockage Hybride :} 
    \begin{itemize}
        \item \textbf{Cache Layer :} Utilisation d'une base en mémoire (type Redis) pour les sessions et les données fréquemment accédées (Popular Content).
        \item \textbf{Search Index :} Un moteur dédié (type Elasticsearch) gère les recherches textuelles complexes et l'indexation des romans.
        \item \textbf{Object Storage :} Les fichiers volumineux (avatars, couvertures) sont isolés sur un stockage objet pour ne pas saturer la base de données relationnelle principale.
    \end{itemize}
\end{itemize}

\subsection{Diagramme d'Objets}

\begin{figure}[H]
    \centering
    \includegraphics[width=1\linewidth]{object diagram.png}
    \caption{Diagramme d'objet}
    \label{fig:placeholder}
\end{figure}

Ce diagramme présente une instance concrète du système à un instant donné, illustrant un auteur (alice\_writer) qui a créé un roman fantasy (myNovel:Novel) avec trois chapitres, tandis qu'un lecteur (bob\_reader) a ajouté ce roman à sa bibliothèque, s'y est abonné, et a publié une critique 5 étoiles. Cette vue complète l'abstraction du diagramme de classes en montrant des valeurs réelles et l'état runtime des objets.

\section{Vue Dynamique du Système}

\subsection{Diagramme d'États}

\begin{figure}[H]
    \centering
    \includegraphics[width=1\linewidth]{state diagram.png}
    \caption{Diagramme d'états}
    \label{fig:placeholder}
\end{figure}

Ce diagramme modélise le cycle de vie complet d'un roman à travers sept états principaux. Un roman naît en état \texttt{Draft} (invisible, éditable librement), passe à \texttt{Active} lors de la publication du premier chapitre (visible dans les recherches, notifications activées), peut être mis en \texttt{Hiatus} (pause temporaire, visible mais inactif), ou marqué \texttt{Completed} (terminé, archivé).

\textbf{Transitions de modération :} les modérateurs peuvent appliquer \texttt{Shadowbanned} (masqué des recherches publiques mais accessible par lien direct) en cas de violation mineure, ou \texttt{Deleted} pour suppression définitive. Des transitions de restauration existent depuis \texttt{Shadowbanned} vers les états précédents après révision.

\textbf{Choix de conception :} l'état \texttt{Shadowbanned} distinct de \texttt{Deleted} permet une modération progressive sans perte irréversible de contenu, facilitant la résolution de conflits et les appels.

\subsection{Diagrammes d'Activité}

\subsubsection{Lecture et Interactions}

\begin{figure}[H]
    \centering
    \includegraphics[width=0.7\linewidth]{activity diagram reading.png}
    \caption{Diagramme d'activité : lecture}
    \label{fig:placeholder}
\end{figure}

Ce workflow modélise le parcours complet d'un utilisateur depuis la recherche d'un roman jusqu'aux interactions post-lecture. Après consultation du synopsis, l'utilisateur lit les chapitres séquentiellement avec mise à jour automatique de sa progression. Les interactions (ajout à la bibliothèque avec activation des notifications, publication de critiques avec recalcul de moyenne) sont conditionnées à l'authentification, avec proposition d'inscription sinon.

\subsubsection{Ecriture}

\begin{figure}[H]
    \centering
    \includegraphics[width=0.33\linewidth]{activity diagram author workflow.png}
    \caption{Diagramme d'activité : écriture}
    \label{fig:placeholder}
\end{figure}

Ce diagramme détaillé couvre cinq partitions : gestion du roman (création avec métadonnées, sélection de genres/tags, initialisation en DRAFT), rédaction de chapitre (édition parallèle du titre, vérification du nombre de mots, prévisualisation), publication (vérification anti-spam, passage en ACTIVE pour le premier chapitre, notifications aux abonnés), suivi analytique (vues, notes, abonnés, reviews), et gestion du statut (transitions vers HIATUS ou COMPLETED).

La vérification anti-spam intervient avant publication pour bloquer le contenu problématique sans impacter les statistiques du roman ni spammer les abonnés.

\subsubsection{Recherche et Découverte}

\begin{figure}[H]
    \centering
    \includegraphics[width=0.95\linewidth]{activity diagram search.png}
    \caption{Diagramme d'activité : recherche}
    \label{fig:placeholder}
\end{figure}

Workflow de recherche avec filtrage multi-critères (genre, tags, statut, note minimale), tri des résultats par pertinence ou popularité, et sauvegarde optionnelle des critères de recherche pour les utilisateurs authentifiés.

\subsection{Diagrammes de Séquence}

\subsubsection{Publication de Chapitre}

\begin{figure}[H]
    \centering
    \includegraphics[width=1\linewidth]{sequence diagram chapter publication.png}
    \caption{Séquence : publication}
    \label{fig:placeholder}
\end{figure}

Séquence détaillée montrant l'interaction entre l'auteur, l'UI, l'objet Novel, la création dynamique d'un objet Chapter, la persistance en base, le recalcul de la note moyenne du roman, et la notification asynchrone des followers via le système de notifications qui interroge la base pour récupérer la liste des abonnés.

\subsubsection{Soumission de Critique}

\begin{figure}[H]
    \centering
    \includegraphics[width=1\linewidth]{sequence diagram review submission.png}
    \caption{Séquence : Review}
    \label{fig:placeholder}
\end{figure}

Workflow montrant qu'un lecteur peut critiquer d'autres romans (héritage Reader $\leftarrow$ Author). La séquence crée un objet Review, le sauvegarde, puis déclenche le recalcul de la moyenne du roman par agrégation (totalStars / reviewCount) avec mise à jour en base de donnée et affichage immédiat.

\subsubsection{Authentification}

\begin{figure}[H]
    \centering
    \includegraphics[width=1\linewidth]{sequence diagram authentication.png}
    \caption{Séquence : Authentification}
    \label{fig:placeholder}
\end{figure}

Processus complet d'inscription (validation email, hachage du mot de passe, envoi email de vérification) et de connexion (vérification logins, création de session, génération de token).

\subsubsection{Gestion de Bibliothèque}

\begin{figure}[H]
    \centering
    \includegraphics[width=1\linewidth]{adding novel to library.png}
    \caption{Séquence : Bibliothèque}
    \label{fig:placeholder}
\end{figure}

Ajout d'un roman à la bibliothèque personnelle du lecteur avec création d'une entrée Library, initialisation de la progression de lecture, et activation optionnelle des notifications de mise à jour.

\subsubsection{Actions de Modération}

\begin{figure}[H]
    \centering
    \includegraphics[width=1\linewidth]{sequence diagram moderation.png}
    \caption{Séquence : modération}
    \label{fig:placeholder}
\end{figure}

Séquence montrant un modérateur récupérant la liste des signalements puis supprimant ou réalisant un shadowban sur un roman problématique avec suppression ou mise à jour de la base de donnée et notification de l'action à l'auteur du roman.

\section{Conclusion}

Cette modélisation UML fournit une spécification complète et rigoureuse d'une plateforme webnovel, couvrant les aspects statiques (structure du domaine, architecture logicielle, infrastructure) et dynamiques (workflows utilisateur, séquences d'interaction, transitions d'états).

Les choix de conception majeurs (héritage Author $\leftarrow$ Reader, composition Novel-Chapter, état Shadowbanned intermédiaire, architecture en couches avec microservices, réplication master-slave, anti-spam en infrastructure) sont tous justifiés par des considérations de maintenabilité, performance, sécurité et expérience utilisateur.

Le projet respecte strictement la norme UML et fournit suffisamment de détails pour permettre de démarrer une implémentation directe dans n'importe quel langage orienté objet. Les diagrammes sont versionnés en PlantUML pour faciliter la collaboration et l'évolution de la documentation.

\end{document}
